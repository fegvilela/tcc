\chapter{Sobre Tipografia}
\label{ch:Tipografia}

\section{Tipografia}

tipografia blablablabla

Tipografia é comunicação escrita por meio de tipos (BIERUT, M., HELFAND, J., HELLER, S. AND POYNOR, R., 2010) surgiu na Alemanha na década de 1450, quando aliaram-se os tipos móveis de metal à prensa resultando no primeiro livro impresso (BRINGHURST, 2011).  A diferença entre a letra escrita e a tipográfica é o método como são geradas, independente do resultado final aparente (Smeijers, 2011). Por exemplo, existem fontes cursivas. Na era digital os tipos móveis foram adaptados para o que chamamos de fontes.


as classificações tipográficas surgiram com a intenção de  criar uniformidade na definição e descrição dos tipos. Ainda que não exista uma classificação oficial na área da tipografia, a classificação criada por maximilien vox e adaptada posteriormente pela ATypl, é uma das classificações mais reconhecidas e utilizadas. Ela une características históricas e estéticas e é dividida em grupos e subgrupos(ROCHA, 2012).
