%%%%%%%%%%%%%%%%%%%%%%%%%%%%%%%%%%%%%%%%%%%%%%%%%%%%%%%%%%%%%%%%%%%%%%%%%
% Agradecimentos
%%%%%%%%%%%%%%%%%%%%%%%%%%%%%%%%%%%%%%%%%%%%%%%%%%%%%%%%%%%%%%%%%%%%%%%%%
% Texto de agradecimentos do primeiro autor
\agradecimentosautori{	``Sê minha vida, sê minha visão''. Esse é o trecho de uma das minhas músicas favoritas. É curioso como ela diz, em poucas palavras, o que a vida de um cristão é e qual deve ser o desejo mais profundo de seu ser. Nesse aspecto, digo que minha vida é Cristo e, por isso, Ele é minha visão em tudo aquilo que faço, mesmo que por vezes eu não reconheça isso. Não é diferente com este projeto. O Deus triúno é criador e mantenedor de todo o universo, portanto, de toda ciência e saber. Os padrões, tema central neste projeto, só existem por isso. A Ele, o mais profundo agradecimento pela vida, pela visão.

	Em segundo lugar, agradeço à minha família. Só eles sabem o que é me aturar de verdade, coitados. Obrigada de coração por todos os dias de apoio amoroso, companhia e desventuras em série. Obrigada, em especial, pelo apoio nestes tempos de TCC, por todas as louças lavadas em meu lugar e tal. Sem vocês, realmente, seria quase impossível viver nesse mundo cinza, mãe, pai, menina, Dedé e Bruno. Um agradecimento agregado ao menino Guilhermão, que me ouviu chorar algumas vezes por causa dos códigos. Talvez não pareça, mas amo vocês.

	À minha (outra) família, meus amados irmãos e melhores amigos: Lu, Vanessa, Kim, Pedrinho, Gigi e Péricles, meu gigante agradecimento. A benção de uma amizade profunda, alegre e edificante tenho em vocês. Obrigada por todos os momentos de broncas, partilhar do coração, diversão e chatice que vocês me proporcionaram durante meu curso e durante o desenvolvimento deste trabalho. Um agradecimento em especial à Lu, menina sonhadora que idealizou o projeto, me botou nessa e me ajudou em vários pontos do desenvolvimento. Obrigada, Vanessinha e Péricles, por me entenderem durante as crises engenheirísticas e por me ajudarem com o texto e tudo o mais. Obrigada, Gigi, por corrigir meu texto mil vezes, haja paciência.

	Muito obrigada, prof. Mylène. Em primeiro lugar por aceitar me orientar mesmo sendo uma ideia meio fora do normal, mesmo não sendo um projeto seu. Obrigada por todo o apoio, paciência, ensinamentos e revisão. Mais do que isso, obrigada por ter sempre me tratado muito bem, deveriam existir mais professores assim.

	Obrigada a todos os meus amigos da elétrica, guerreiros! Foi um prazer ter a amizade de vocês durante os anos de muita luta. É ótimo saber que, para alguns desses, a amizade é muito mais do que circunstancial. Wawa e Pedro, vocês são meus dois melhores amigos que fiz nessa época, fico feliz porque a parceria é sólida e prazerosa. Dani, Aline, Natasha, Helô, Raquel, Iago e Rodrigo, obrigada por todo o tempo juntos, por vários dias felizes, por outros tristes, por muito cálculo, café e parceria.

	Um agradecimento final por todos aqueles que, de uma forma ou de outra, estiveram presentes em minha formação! Agradeço ao Marcos Mourthè, Calil, Pri, Bruno, Kukas, Donatos, Mona, Yam, família Cruz, primos e tios!
 }

% Texto de agradecimentos do segundo autor. Caso não tenha um segundo autor, este texto não
% será mostrado
\agradecimentosautorii{A inclusão desta seção de agradecimentos é opcional e fica à critério do(s) autor(es), que caso deseje(em) inclui-la deverá(ao) utilizar este espaço, seguindo está formatação.}

% Texto de agradecimentos do segundo autor. Caso não tenha um terceiro autor, este texto não
% será mostrado
\agradecimentosautoriii{A inclusão desta seção de agradecimentos é opcional e fica à critério do(s) autor(es), que caso deseje(em) inclui-la deverá(ao) utilizar este espaço, seguindo está formatação.}
