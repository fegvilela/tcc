%TCIDATA{LaTeXparent=0,0,relatorio.tex}

\resumo{Resumo}{O deficiente visual vivencia exclusão em variados aspectos, tanto social, quanto em esferas educacionais e culturais. Isto pode ser exemplificado pelo fato de o deficiente visual encontrar várias barreiras em relação à ambientes físicos que não são adaptados a eles e também ao acesso restrito a livros e textos. Além disso, percebe-se uma distanciação dessas pessoas em relação às expressões artísticas variadas, como teatro, museus e filmes. Dentro deste contexto, insere-se também a tipografia. Além de se aproximar da escrita como uma forma de comunicação, a tipografia também comunica por meio de uma linguagem visual ligada à estética. Sendo assim, um produto de tecnologia assistiva para ensino de tipografia a deficientes visuais foi proposto, em trabalho anterior, como forma de diminuir a exclusão do deficiente visual com o campo da tipografia, aproximando-o também de aspectos culturais nos quais a tipografia é aplicada, como marcas de carros, de filmes, de bandas, entre outros. Desta forma, este trabalho descreve o desenvolvimento de uma parte deste produto de tecnologia assistiva, o sistema de reconhecimento de padrões em tipos, que irá compor o software auxiliar ao deficiente visual. Na etapa de desenvolvimento do algoritmo, foram utilizadas técnicas de aprendizado de máquina (\textit{Machine Learning}), no qual foi aplicado o operador Padrão Binário Local (LBP, \textit{Local Binary Pattern} em inglês) para extração de atributos e os classificadores Máquina de Vetor de Suporte (SVM, em inglês \textit{Support Vector Machine}) e Floresta Aleatória (\textit{Random Forest Classifier}). O problema neste projeto é caracterizado como classificação e, como resultado, obteve-se uma índice de acerto da classificação de 84,91\% no melhor caso.}

\vspace*{2cm}

\resumo{Abstract}{The visual impaired experiences exclusion in various aspects, both social, as well as in educational and cultural spheres. This can be exemplified by the fact that the visual impaired finds various barriers to physical environments that are not adapted to them and also restricted access to books and texts. In addition, there is a gap between them and varied artistic expressions, such as theater, museums and films. In this context, typography is also included. Besides a form of communication as writing is, typography also communicates through a visual language linked to aesthetics. Thus, an assistive technology product for teaching typography to the visually impaired was proposed, in a previous work, as a way to reduce the exclusion of the visually impaired towards the field of typography, also approaching cultural aspects in which the typography is applied, such as logotypes of cars, films, bands and others. In this way, this work describes the development of a part of this assistive technology product, the system of pattern recognition in types, that will compose the auxiliary software for the visually impaired. In the algorithm development stage, Machine Learning techniques were used, applying the Local Binary Pattern (LBP) operator for feature extraction and the classifiers Support Vector Machine (SVM) and Random Forest Classifier. The problem presented in this project is characterized as classification and the result accuracy obtaind was 84.91\%, in the best performance.}
