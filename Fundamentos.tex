\chapter{Fundamentos Teóricos}

\section{Tipografia}

tipografia blablablabla

\section{Reconhecimento de Padrões empregando Machine Learning}


\subsection{Conceitos Gerais}

\subsection{Extração de Atributos (Features)}

O método usado foi Local Binary Pattern.

Considerações sobre dimensionalidade da imagem.

\subsubsection{Local Binary Pattern}


\subsection{Classificadores}

Os modelos usados para classificação no sistema foram: Máquina de Vetor de Suporte (SVM, em inglês \textit{Support Vector Machine}) e Floresta Aleatória (\textit{Random Forest Classifier}).

\subsubsection{Máquina de Vetor de Suporte}

As Máquinas de Vetor de Suporte são um modelo de aprendizado que vem crescendo há um tempo na comunidade de Aprendizado de Máquina e é utilizado em variadas aplicações, desde bioinformática a reconhecimento de imagens, como o caso desse projeto \citeC{mitchell1997} \citeC{noble2004} \citeC{kim2002}.

Intrinsicamente, o SVM é um modelo de classificação binária em problemas de conhecimento controlado (\textit{supervised learning}). Concedido um conjunto de dados para treinamento, todos


%Given a set of training examples, each marked as belonging to one or the other of two categories, an SVM training algorithm builds a model that assigns new examples to one category or the other, making it a non-probabilistic binary linear classifier (although methods such as Platt scaling exist to use SVM in a probabilistic classification setting)
\subsubsection{Floresta Aleatória}
